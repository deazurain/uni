\documentclass[11pt]{article}
\usepackage{fullpage}
\usepackage{algorithm2e}

\title{I2300 Algoritmiek\\Homework Exercise 2a}

\author{\begin{tabular}{l|l}Mick van Gelderen&409 1566\end{tabular}}

\date{November 2011}

\begin{document}

\maketitle

\vspace{10mm}

\section*{Legal}
I have discussed the excercises with Jelle Licht and Felix Akkerman. 

\section{}
The optimal solution for this particular problem is contestant 2 and then contestant 1, this gives a total time of $4 + 6 + 3 + 2 = 15$.

\section{}
Asuming there are one or more weights and none of the weights exceed the maximum a truck can handle. 

\vspace{10pt}

\begin{algorithm}[H]
\label{alg1}
\caption{Calculate number of trucks required}
\SetLine
running times $r_{i}, ..., r_{n}$\;
biking times $b_{i}, ..., b_{n}$\;
swimming times $w_{i}, ..., w_{n}$\;
$C$\tcc*{a list of contestant numbers from 1 to n}
$S$\tcc*{a list of contestant starting times}
\vspace{2pt}
Sort $C$ by $r_{i} + b_{i} \le r_{i+1} + b_{i+1}$\;
$S_{0} = 0$\;
$S_{1} = w_{C_{1}}$\;
\ForEach(\tcc*[f]{i becomes the number of the current contestant}){$C$ as $i$}{
  $S_{i} \gets S_{i - 1} + w_{C_{i}}$\tcc*{set current contestants' start time to the last start time + his swim time}
}

\end{algorithm}

\section{}
The main loop of the algorithm is $O(n)$ but the sorting is $O(n\log{n})$. The sorting outweighs $O(n)$ so the algorithm has a tightest worst-case upper bound of $O(n\log{n})$. 

\section{}
Too tired...

\end{document}


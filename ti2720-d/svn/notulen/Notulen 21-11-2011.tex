\documentclass[12pt]{article}
\setlength{\parindent}{0mm}
\usepackage[margin=1in]{geometry}
\begin{document}
\pagestyle{empty}

\huge
Notulen\\
\small
21 November 2011 \\
Groep 6 \\\\
\normalsize
\begin{tabular}{rl}
Voorzitter & Mick van Gelderen \\
Notulist & Jeroen Bareman \\
Deelnemers & Jeroen Bareman, Mick van Gelderen, Luyt Visser, \\
 & Jelle Licht, Niels \\\\
\end{tabular} 
\large
\\Actiepunten \\\\
\normalsize
1. Opening\\
2. Aanwezigheid
\begin{quote}
  Roelof Sol is afwezig door ziekte.
\end {quote}
3. Agenda vaststellen 
\begin{quote}
  We hebben besloten dat punt 6, het ontwerp, wordt verschoven naar 28 november, omdat dit op dit moment nog geen prioriteit heeft. 
\end {quote}
4. Actiepunten bespreken 
\begin{quote}
  Notulen schrijven m.b.v. latex kan door "sudo apt get-install texlife full" te installeren en vervolgens het commando: "pdflatex notulen.tex" uit te voeren.
\end {quote}
5. Sensor plots 
\begin{quote}
  Communiceren met de robot lukt nog niet, Jeroen en Luyt gaan dit na de vergadering verder uitzoeken.
\end {quote}
6. Ontwerp \\
7. Feedback haalbaarheid 
\begin{quote}
  We kregen feedback van de studentassistent. De studentassistent miste in onze haalbaarheidsstudie het onderwerp motorbesturing, de motor maakt gebruik van PWM en we moeten uitleggen hoe we dit willen gebruiken. In de haalbaarheidsstudie hadden we ook de minimale breedte van de gang niet berekend waar de robot nog doorheen zou moeten passen. Over het algemeen gezien was de haalbaarheidsstudie te vaaag en moest hij concreter. Het laatste punt van feedback was dat een nieuwe alinea moest beginnen met een tab en niet alleen een enter voor het eind van de regel bij de vorige alinea. We hebben besloten dat Mick en Jella na de vergadering de feedback zullen verwerken en een nieuwe verbeterde versie van de haalbaarheidsstudie zullen inleveren.
\end {quote}
8. W.V.T.T.K. \\
9. Nieuwe actiepunten \\ 
10. Datum volgende vergadering 
\begin{quote}
  Op maandag 28 november 09:00 zal de volgende vergadering plaatsvinden.
\end {quote}
11. Sluiting \\
\end{document}
